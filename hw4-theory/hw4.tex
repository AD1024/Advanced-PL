\documentclass{article}

\usepackage{lmodern}
\usepackage{concrete}
\usepackage{eulervm}
\usepackage[T1]{fontenc}
\usepackage{amsmath,amsthm,amssymb}
\usepackage{mathpartir}
\usepackage{stmaryrd}
\usepackage{xcolor}
\usepackage[shortlabels]{enumitem}
\usepackage{listings}

% copied from package braket
% edited to remove whitespace inside braces
% and to simplify the definition of \Set because we don't use double-vert
{\catcode`\|=\active
  \xdef\set{\protect\expandafter\noexpand\csname set \endcsname}
  \expandafter\gdef\csname set \endcsname#1{\mathinner
    {\lbrace{\mathcode`\|32768\let|\midvert #1}\rbrace}}
  \xdef\Set{\protect\expandafter\noexpand\csname Set \endcsname}
  \expandafter\gdef\csname Set \endcsname#1{\left\{%
     {\mathcode`\|32768\let|\SetVert #1}\right\}}
}
\def\midvert{\egroup\mid\bgroup}
\makeatletter
\def\mid@vertical{\mskip1mu\vrule\mskip1mu}
\def\SetVert{{\egroup\;\mid@vertical\;\bgroup}}
\makeatother


\lstdefinelanguage{imp}
{morekeywords={while,assert,skip,failed},
sensitive=true,
morecomment=[l]{//},
morestring=[b]",
}
\lstset{
language=imp,
basicstyle=\ttfamily,
keywordstyle=\bfseries,
identifierstyle=,
showstringspaces=false}


\theoremstyle{definition}
\newtheorem{theorem}{Theorem}

\usepackage[margin=1in]{geometry}
\usepackage{changepage}

\usepackage{fancyhdr}
\pagestyle{fancy}
\fancyhead[C]{Homework 4: Theory part}
\fancyhead[R]{CSE 490P}
\setlength{\headheight}{14pt}
\newcommand{\meta}[1]{{\color{blue}#1}}
\newcommand{\todo}[1]{{\huge\color{red}#1}}
\newcommand{\imp}{\textsc{Imp}}
\newcommand{\stlc}{\textsc{STLC}}
\newcommand{\dom}[1]{\ensuremath{\text{dom}\ #1}}

%\skip is already a LaTeX command, so we pick another name
\newcommand{\progtext}[1]{\ensuremath{\text{\ttfamily\bfseries #1}}}
\newcommand{\impskip}{\progtext{skip}}
\newcommand{\assert}{\progtext{assert}}
\newcommand{\failed}{\progtext{failed}}
\newcommand{\while}{\progtext{while}}
\newcommand{\assign}{\ensuremath{\mathrel{\text{\texttt{:=}}}}}
\newcommand{\progif}{\progtext{if}}
\newcommand{\progthen}{\progtext{then}}
\newcommand{\progelse}{\progtext{else}}
\newcommand{\progtrue}{\progtext{true}}
\newcommand{\progfalse}{\progtext{false}}

\begin{document}
\noindent{}The last page of this document contains a reference for \imp{}'s operational semantics and its Hoare-style logic.
\begin{enumerate}[start=1,label={{\bf Problem \arabic*}.},ref=\arabic*,left=0pt..0pt,widest*=10,align=left,itemindent=*]
\item This problem covers the $\while$ case in the proof of soundness for Hoare
  logic (Theorem~\ref{thm:hoare-sound}). Remember that the proof was by
  induction on the derivation of $\set{\meta{P}}\meta{s}\set{\meta{Q}}$.
  \begin{enumerate}[(a),left=1em]
  \item Write out the statement of exactly what we need to prove in the case for the $\while$ rule.
  
    \textbf{Show that if $\set{\meta{I}}\ \while\ \meta{e}\ \meta{s}\ \set{\meta{I}\wedge\meta{\neg e}}$ and $\meta{H}\models \meta{I}$ and $\meta{H},\while\ \meta{e}\ \meta{s} \to^* \meta{H^\prime},\meta{s^\prime}$ then $\meta{s^\prime}\neq \failed$ and if $\meta{s^\prime}=\impskip$ then $\meta{H^\prime} \models \meta{I}\wedge \meta{\neg e}$}

  \item Write out the statement of the induction hypothesis we get from rule induction.
  
  \textbf{Induction Hypothesis: if $\set{\meta{I}\wedge \meta{e}}\meta{s}\set{\meta{I}}$ and $\meta{H}\models\meta{\meta{I}\wedge \meta{e}}$ and $\meta{H},\meta{s}\to^* \meta{H^\prime},\meta{s^\prime}$ then $\meta{s}\neq\failed$ and if $\meta{s^\prime}=\impskip$ then $\meta{H^\prime} \models 
  \meta{I}$}

  \item Consider the special case where $\meta{s'} = \impskip$. Explain informally in one or two sentences what an execution
    \[
      \meta{H}, \while\ \meta{e}\ \meta{s} \to^* \meta{H'},\impskip
    \]
    looks like by relating the execution of the loop to the execution of its body $\meta{s}$.

    Hint: If you're not sure how to proceed, check out the picture
    on slide 8 in lecture 12 and see if you can figure out what I meant there.

    \begin{itemize}
      \item Case 1. $\meta{H}, \meta{e} \Downarrow \bot$. Then by the operational semantics, $\meta{H}, \while\ \meta{e}\ \meta{s} \to^* \meta{H}, \impskip$
      \item Case 2. $\meta{H}, \meta{e} \Downarrow \top$. Then $\meta{s}$ is executed by at least once, until the conditional $\meta{e}$ evaluated to $\bot$, the whole expression steps to $\meta{H^\prime}, \impskip$.
    \end{itemize}

  \item\label{prob:while-to-skip-lemma} State (but do not prove) a formal lemma for this special case of the form
    \begin{quote}
      If $\meta{H}, \while\ \meta{e}\ \meta{s} \to^* \meta{H'},\impskip$, then ...
    \end{quote}
    In order to state your lemma, you will very likely need to introduce a new
    concept/definition.  State clearly what concept you're introduce and define it
    precisely. Again, do not attempt to prove your lemma. (It's not going to be
    strong enough to prove, and it's not sufficient to finish the soundness proof anyway.)

    \textbf{Define $\meta{H}\to_s^* \meta{H^\prime}$ denotes that the initial heap is $H$, and after executing $\meta{s}$ zero or more times, the resulted heap is $\meta{H^\prime}$}

    \textbf{Lemma 1-1. $\meta{H}, \while\ \meta{e}\ \meta{s} \to^* \meta{H'},\impskip$, then $\exists \meta{H_0}, \meta{H}\to_s^* \meta{H_0}$ and $\meta{H^\prime} = \meta{H_0}$}.

  \item Now consider the general case where $\meta{s'}$ can be anything. Explain informally in one or two sentences
    what an execution
    \[
      \meta{H}, \while\ \meta{e}\ \meta{s} \to^* \meta{H'},\meta{s'}
    \]
    looks like by relating it to the execution of the loop body $\meta{s}$.

    Hint: The global structure is similar to the special case where
    $\meta{s'}=\impskip$. The only difference is on the last iteration of the
    loop. The picture on slide 8 of lecture 12 may help again.

    \begin{itemize}
      \item Case. $\meta{s^\prime}=\impskip$. Case above.
      \item Case. $\meta{s^\prime}=\failed$. This must be the case where $\meta{H}, \meta{e} \Downarrow \top$. Moreover, the only possibility is that $\meta{H}, \meta{s} \to^* \meta{H_1}, \meta{s}$ and then $\meta{H_1}, \meta{s} \to \meta{H^\prime}, \failed$.
      \item Case. $\meta{s^\prime}$ is in the middle of the execution of the while loop.$\exists\ \meta{s_1}$ such that $\meta{H}, \meta{s} \to^* \meta{H^\prime}, \meta{s_1}$ under the condition that $\meta{H}, \meta{e} \Downarrow \top$
      \item Case. $\while\ \meta{e}\ \meta{s}$. Takes zero step, so $\meta{H^\prime}=\meta{H}$. 
    \end{itemize}

  \item State (but do not prove yet) a formal lemma for this special case of the form
    \begin{quote}
      If $\meta{H}, \while\ \meta{e}\ \meta{s} \to^* \meta{H'},\meta{s'}$, then ...
    \end{quote}
    You will likely need to use the same concept you introduced in part \ref{prob:while-to-skip-lemma}.
    The scribblings on slide 10 of lecture 12 may help you.

    \textbf{Lemma 1-2. If $\meta{H}, \while\ \meta{e}\ \meta{s} \to^* \meta{H'},\meta{s'}$, then $\exists \meta{H_1}. \meta{H}\to_{\meta{s}}^* \meta{H_1}$ and one of the following holds}
    \begin{itemize}
      \item Case $\meta{s^\prime}=\while\ \meta{e}\ \meta{s}$. $\meta{H_1}=\meta{H^\prime}$
      \item Case $\meta{s^\prime}=\impskip$. $\meta{H_1}=\meta{H^\prime}$ and $\meta{H^\prime}, \meta{e}\Downarrow \bot$
      \item Case $\meta{s^\prime}=\failed$. $\meta{H_1}, \meta{s} \to \meta{H^\prime}, \failed$ and $\meta{H_1}, \meta{e} \Downarrow \top$.
      \item Case $\meta{s^\prime}=\meta{s_1}; \while\ \meta{e}\ \meta{s}$ for some $\meta{s_1}$. $\meta{H_1}, \meta{e} \Downarrow \top$ and $\meta{H_1}, \meta{s} \to^* \meta{H^\prime}, \meta{s_1}$.
    \end{itemize}

  \item Remember that we should never attempt to prove a lemma unless we are
    sure it help us with our bigger proof. Complete the proof of the $\while$
    case of the soundness theorem using your lemma.
    \begin{proof}
      While case soundness.
      \begin{itemize}
        \item Case. $\inferrule{
          \set{\meta{I}\wedge \meta{e}} \meta{s} \set{\meta{I}}
        }{\set{\meta{I}} \while\ \meta{e}\ \meta{s} \set{\meta{I} \wedge \meta{\neg e}}}$. By applying the lemma, we split to four cases:
        \begin{enumerate}
          \item $\meta{s^\prime}=\impskip$. According to induction hypothesis, we have $\meta{H^\prime}\models \meta{I}$. Also, we know, by case spliting, that $\meta{H^\prime}, \meta{e} \Downarrow \bot$; therefore, since $\meta{H^\prime}$ is not empty and $\meta{H^\prime} \not\models \meta{e}$, thus $\meta{H^\prime}\models\meta{\neg e}$. Therefore, $\meta{H^\prime} \models \set{\meta{I} \wedge \meta{\neg e}}$.
          \item $\meta{s^\prime}=\while\ \meta{e}\ \meta{s}$. Then since $\meta{s^\prime}$ is not $\failed$, and $\meta{s^\prime}$ is not $\impskip$, the conclusion holds trivially.
          \item $\meta{s^\prime}=\failed$. According to the condition on case split, $\exists\ \meta{H_1}, \meta{s} \to \meta{H^\prime}, \failed$. However, according to our induction hypothesis, for any $\meta{H} \models \set{\meta{I}\wedge \meta{e}}$, $\meta{H}, \meta{s} \to^* \meta{H_1^\prime}, \meta{s^\prime}$, then $\meta{s^\prime}$ is not $\failed$, which contradicts to our assumption. Thus the conclusion holds trivially.
          \item $\meta{s^\prime}$ is in the middle of the execution. Similar as Case ii.
        \end{enumerate}
      \end{itemize}
    \end{proof}
  \item Prove your lemma by rule induction on the derivation $\to^*$. Be sure to
    use the definition of $\to^*$ included on the last page of this document,
    and not the one in the slides from week 2, or your life will be miserable.
    \begin{proof}
      By induction on $\to^*$.
      \begin{itemize}
        \item $\inferrule{ }{\meta{H}, \while\ \meta{e}\ \meta{s} \to^* \meta{H}, \while\ \meta{e}\ \meta{s}}$. Pick $\meta{H^\prime}=\meta{H}$, and Case 1 in \textbf{Lemma 1-2} holds. 
        \item $\inferrule{
          \meta{H}, \while\ \meta{e}\ \meta{s} \to \meta{H_1}, \meta{s_1}\and
          \meta{H_1}, \meta{s_1} \to^* \meta{H^\prime}, \meta{s^\prime}
        }{\meta{H_1}\while\ \meta{e}\ \meta{s}\to^* \meta{H^\prime}, \meta{s^\prime}}$. By case analysis on $\meta{s_1}$
        \begin{enumerate}
          \item Case $\meta{s_1} = \meta{s}; \while\ \meta{e}\ \meta{s}$. In this case the next steps are $$\inferrule{
            \meta{H_1}, \meta{s_1};\while\ \meta{e}\ \meta{s} \to \meta{H_1^\prime}, \meta{s_1^\prime}; \while\ \meta{e}\ \meta{s}\and
            \meta{H_1^\prime}, \meta{s_1^\prime};\while\ \meta{e}\ \meta{s} \to^* \meta{H^\prime}, \meta{s^\prime}
          }{\meta{H_1}, \meta{s_1};\while\ \meta{e}\ \meta{s}\to^* \meta{H^\prime}, \meta{s^\prime}}$$ with $\meta{H_1}, \meta{s_1}\to \meta{H_1^\prime}, \meta{s_1^\prime}$. By cases analysis on $\meta{s_1^\prime}$
          \begin{enumerate}
            \item Case $\meta{s_1^\prime}$ is not $\failed$. In this case, since we are to execute the body of while loop, according to the definition of $\to$ on $\while$, $\meta{H_1}, \meta{e} \Downarrow \top$. Since, according to the assumptions, $\meta{H_1}, \meta{s_1} \to^* \meta{H^\prime}, \meta{s^\prime}$, Case 4 of \textbf{Lemma 1-2} holds.
            \item Case $\meta{s_1^\prime}$ is $\failed$. Then we know that $\meta{H_1}, \meta{s_1}\to \meta{H_1^\prime}, \failed$
          \end{enumerate}
          \item Case $\meta{s_1}=\impskip$. Then $\meta{H_1}, \meta{e} \Downarrow \bot$ according to the definition of $\to$ on $\while$. Then, since $\meta{s_1}=\impskip$, $\meta{H_1}, \impskip \to^* \meta{H^\prime}, \impskip$, where $\meta{H^\prime}=\meta{H_1}$, therefore, Case 2 of $\textbf{Lemma 1-2}$ holds.
          \item Case $\meta{s_1} = \failed$. Impossible.
        \end{enumerate}
      \end{itemize}
    \end{proof}
  \end{enumerate}
  The remaining subproblems are extra credit.
  \begin{enumerate}[resume*]
  \item \textit{Extra credit.} Explain informally in one sentence the difference between
    the definition of $\to^*$ in this document and the one we used in week 2.
  \item \textit{Extra credit.} Explain informally in one or two sentences why your lemma
    would not be directly provable on the definition of $\to^*$ from week 2.
  \item \textit{Extra credit.} Prove (by induction on various things) that the
    two definitions of $\to^*$ are equivalent in the sense that they relate the
    same heap-statement pairs to each other. Be sure to make it clear which
    definition you are referring to at any given time, perhaps by giving them
    different names like $\to_1^*$ and $\to^*_2$ or something. In our solution,
    we needed two top-level inductions (one for each direction) and two lemmas
    proved by induction (one for each direction).
  \end{enumerate}
\end{enumerate}

\clearpage
  \noindent\imp
  \[
    \begin{array}{rcl}
      \meta{e} & ::= & \meta{n} \mid \meta{e} + \meta{e} \mid \meta{e} - \meta{e} \mid \meta{e} \wedge \meta{e} \mid \lnot \meta{e}
                       \mid \meta{b} \mid \meta{e} < \meta{e} \mid \meta{e} = \meta{e} \mid \meta{x}\\
      \meta{v} & ::= & \meta{n} \mid \meta{b}\\
      \meta{s} & ::= & \impskip \mid \assert\ \meta{e} \mid \failed \mid \meta{x} \assign \meta{e} \mid \meta{s}; \meta{s} \mid \while\ \meta{e}\ \meta{s}\\
      \meta{\tau} & ::= & int \mid bool
    \end{array}
    \qquad
    \begin{array}{rcl}
      \meta{n} & \in & \mathbb{Z}\\
      \meta{b} & \in & \mathbb{B}\\
      \meta{x} & \in & \text{Var}\ (= \text{String})\\
      \meta{H} & \in & \text{Var} \rightharpoonup \text{Value}\\
      \meta{\Sigma} & \in & \text{Var} \rightharpoonup \text{Type}
    \end{array}
  \]
  \boxed{\meta{H},\meta{s} \to \meta{H},\meta{s}}
  \begin{mathpar}
    \inferrule{\meta{H},\meta{e}\Downarrow\meta{v}}{\meta{H},\meta{x}\assign\meta{e} \to \meta{H}[\meta{x}\mapsto\meta{v}],\impskip}
    \and
    \inferrule{\meta{H},\meta{e}\Downarrow\meta{\top}}{\meta{H},\assert\ \meta{e} \to \meta{H},\impskip}
    \and
    \inferrule{\meta{H},\meta{e}\Downarrow\meta{\bot}}{\meta{H},\assert\ \meta{e} \to \meta{H},\failed}
    \and
    \inferrule{\meta{H},\meta{s_1} \to \meta{H'},\meta{s_1'}}{\meta{H},\meta{s_1};\meta{s_2} \to \meta{H'},\meta{s_1'};\meta{s_2}}
    \and
    \inferrule{ }{\meta{H},\impskip;\meta{s} \to \meta{H},\meta{s}}
    \and
    \inferrule{ }{\meta{H},\failed;\meta{s} \to \meta{H},\failed}\\
    \and
    \inferrule{\meta{H},\meta{e}\Downarrow\meta{\top}}{\meta{H},\while\ \meta{e}\ \meta{s} \to \meta{H},\meta{s};\while\ \meta{e}\ \meta{s}}
    \and
    \inferrule{\meta{H},\meta{e}\Downarrow\meta{\bot}}{\meta{H},\while\ \meta{e}\ \meta{s} \to \meta{H},\impskip}
  \end{mathpar}
  \boxed{\meta{H},\meta{s} \to^* \meta{H},\meta{s}}
  \begin{mathpar}
    \inferrule{ }{\meta{H},\meta{s} \to^* \meta{H},\meta{s}}
    \and
    \inferrule{\meta{H_1},\meta{s_1} \to^* \meta{H_2},\meta{s_2} \and \meta{H_2},\meta{s_2} \to \meta{H_3},\meta{s_3}}{\meta{H_1},\meta{s_1} \to^* \meta{H_3},\meta{s_3}}
  \end{mathpar}
  \noindent\boxed{\set{\meta{P}}\meta{s}\set{\meta{Q}}}
  \begin{mathpar}
    \inferrule{ }{\set{\meta{P}}\impskip\set{\meta{P}}}
    \and
    \inferrule{\meta{P}\Rightarrow\meta{e}}{\set{\meta{P}}\assert\ \meta{e}\set{\meta{P}}}
    \and
    \inferrule{ }{\set{\meta{P}[\meta{e}/\meta{x}]}\meta{x} \assign \meta{e}\set{\meta{P}}}
    \and
    \inferrule{\set{\meta{P}}\meta{s_1}\set{\meta{R}}\and \set{\meta{R}}\meta{s_2}\set{\meta{Q}}}{\set{\meta{P}}\meta{s_1};\meta{s_2}\set{\meta{Q}}}
    \and
    \inferrule{\set{\meta{I}\wedge\meta{e}}\meta{s}\set{\meta{I}}}{\set{\meta{I}}\while\ \meta{e}\ \meta{s}\set{\meta{I}\wedge\lnot\meta{e}}}
    \and
    \inferrule{\meta{P}\Rightarrow \meta{P'}\and\set{\meta{P'}}\meta{s}\set{\meta{Q'}}\and\meta{Q'}\Rightarrow\meta{Q}}{\set{\meta{P}}\meta{s}\set{\meta{Q}}}
  \end{mathpar}
  \noindent\boxed{\meta{H}\vDash \meta{P}}
  \[
    \begin{array}{rcl}
      \meta{H}\vDash \meta{P} & = & \meta{H},\meta{P}\Downarrow \meta{\top}
    \end{array}
  \]
  \begin{theorem}[Soundness of the logic with respect to the operational semantics.]\label{thm:hoare-sound}
    If $\set{\meta{P}}\meta{s}\set{\meta{Q}}$ and $\meta{H}\vDash\meta{P}$ and $\meta{H},\meta{s}\to^*\meta{H'},\meta{s'}$,
    then $\meta{s'}\ne\failed$ and if $\meta{s'}=\impskip$ then $\meta{H'}\vDash\meta{Q}$.
  \end{theorem}
  \begin{proof}
    By induction on the derivation of $\set{\meta{P}}\meta{s}\set{\meta{Q}}$. All cases but the $\while$ rule were covered in lecture.
    (Slides 8 and 10 of Lecture 12 cover the $\while$ rule, but we didn't talk about them in lecture due to time.)
  \end{proof}
\end{document}