\documentclass{article}

\usepackage{lmodern}
\usepackage{concrete}
\usepackage{eulervm}
\usepackage[T1]{fontenc}
\usepackage{amsmath,amsthm,amssymb}
\usepackage{mathpartir}
\usepackage{stmaryrd}
\usepackage{xcolor}
\usepackage[shortlabels]{enumitem}
\usepackage{listings}

% copied from package braket
% edited to remove whitespace inside braces
% and to simplify the definition of \Set because we don't use double-vert
{\catcode`\|=\active
  \xdef\set{\protect\expandafter\noexpand\csname set \endcsname}
  \expandafter\gdef\csname set \endcsname#1{\mathinner
    {\lbrace{\mathcode`\|32768\let|\midvert #1}\rbrace}}
  \xdef\Set{\protect\expandafter\noexpand\csname Set \endcsname}
  \expandafter\gdef\csname Set \endcsname#1{\left\{%
     {\mathcode`\|32768\let|\SetVert #1}\right\}}
}
\def\midvert{\egroup\mid\bgroup}
\makeatletter
\def\mid@vertical{\mskip1mu\vrule\mskip1mu}
\def\SetVert{{\egroup\;\mid@vertical\;\bgroup}}
\makeatother


\lstdefinelanguage{imp}
{morekeywords={while,assert,skip,failed},
sensitive=true,
morecomment=[l]{//},
morestring=[b]",
}
\lstset{
language=imp,
basicstyle=\ttfamily,
keywordstyle=\bfseries,
identifierstyle=,
showstringspaces=false}


\theoremstyle{definition}
\newtheorem{theorem}{Theorem}

\usepackage[margin=1in]{geometry}
\usepackage{changepage}

\usepackage{fancyhdr}
\pagestyle{fancy}
\fancyhead[C]{Homework 5: Theory part}
\fancyhead[R]{CSE 490P}
\setlength{\headheight}{14pt}
\newcommand{\meta}[1]{{\color{blue}#1}}
\newcommand{\todo}[1]{{\huge\color{red}#1}}
\newcommand{\imp}{\textsc{Imp}}
\newcommand{\stlc}{\textsc{STLC}}
\newcommand{\dom}[1]{\ensuremath{\text{dom}\ #1}}

%\skip is already a LaTeX command, so we pick another name
\newcommand{\progtext}[1]{\ensuremath{\text{\ttfamily\bfseries #1}}}
\newcommand{\impskip}{\progtext{skip}}
\newcommand{\assert}{\progtext{assert}}
\newcommand{\failed}{\progtext{failed}}
\newcommand{\while}{\progtext{while}}
\newcommand{\assign}{\ensuremath{\mathrel{\text{\texttt{:=}}}}}
\newcommand{\progif}{\progtext{if}}
\newcommand{\progthen}{\progtext{then}}
\newcommand{\progelse}{\progtext{else}}
\newcommand{\progtrue}{\progtext{true}}
\newcommand{\progfalse}{\progtext{false}}

\begin{document}
\noindent{}The last two pages of this document contain a reference on the syntax, operational semantics, type system, and (unary) logical relation for System F.
\begin{enumerate}[start=1,label={{\bf Problem \arabic*}.},ref=\arabic*,left=0pt..0pt,widest*=10,align=left,itemindent=*]
\item Prove (by induction on a thing of your choice) the following statement about System F's type system.
  \begin{quote}
    If $\cdot;\cdot\vdash\meta{e}:\meta{\tau}$, then $\meta{\tau}$ is closed (that is, $FTV(\meta{\tau}) = \emptyset$).
  \end{quote}

  Hint: This is similar to Problem~1 on Homework~3, except that we are talking about the free \emph{type} variables of the \emph{type} $\meta{\tau}$,
  rather than the free term variables of $\meta{e}$.

  Hint: You will need a similar generalization as you did with Problem~1 on
  Homework~3, except at the type variable level rather than the term variable
  level. There is also an additional complication due to the interaction between
  $\meta{\Gamma}$ and $\meta{\Delta}$ in the typing judgment. Proceed carefully!

  \begin{quote}
    \textbf{Lemma 0. forall $\meta{\Delta}$, and type variables $\meta{\alpha}$, $\meta{\beta}$, if $\meta{\alpha} \neq \meta{\beta}$ and $\meta{\Delta}, \meta{\alpha}\vdash\meta{\beta}$ then $\meta{\Delta}\vdash\meta{\beta}$}.
  \end{quote}
  \begin{proof}
    By induction on $\meta{\Delta},\meta{\alpha}\vdash\meta{\beta}$.
    \begin{itemize}
      \item Case $\inferrule{
        \meta{\beta} \in \meta{\Delta}, \meta{\alpha}
      }{\meta{\Delta},\meta{\alpha}\vdash\meta{\beta}}$. Since $\meta{\beta}\neq\meta{\alpha}$, it must be the case that $\meta{\beta}\in \meta{\Delta}$. Therefore, $\meta{\Delta}\vdash\meta{\beta}$.
      \item Other cases are trivial since $\meta{\beta}$ is not a type variable.
    \end{itemize}
  \end{proof}

  \begin{quote}
      \textbf{Lemma 1-0: forall $\meta{\Delta}$, $\meta{\alpha}$, $\meta{\beta}$, if $\meta{\beta} \in FTV(\meta{\alpha}$) and $\meta{\Delta}\vdash\meta{\alpha}$ then $\meta{\Delta}\vdash\meta{\beta}$}.
  \end{quote}
  \begin{proof}
    By induction on $\meta{\Delta}\vdash\meta{\alpha}$.
    \begin{itemize}
      \item Case $\inferrule{
        \meta{\alpha} \in \meta{\Delta}
      }{\meta{\Delta}\vdash\meta{\alpha}}$. In this case, $FTV(\meta{\alpha})=\emptyset$, therefore this case is vacuous.
      \item Case $\inferrule{
        \meta{\Delta}\vdash\meta{\tau_1}\and
        \meta{\Delta}\vdash\meta{\tau_2}
      }{\meta{\Delta}\vdash\meta{\tau_1}\to\meta{\tau_2}}$. $\meta{\beta}\in FTV(\meta{\tau_1}) \cup FTV(\meta{\tau_2})$, by case split on the occurence of $\meta{\beta}$:
        \begin{enumerate}
          \item $\meta{\beta}\in FTV(\meta{\tau_1})$. According to the induction hypothesis, $\meta{\Delta}\vdash\meta{\beta}$
          \item $\meta{\beta}\in FTV(\meta{\tau_2})$. Similar as previous case.
        \end{enumerate}
      \item Case $\inferrule{
        \meta{\alpha}\not\in \meta{\Delta}\and
        \meta{\Delta},\meta{\alpha}\vdash\meta{\tau}
      }{\meta{\Delta}\vdash\forall \meta{\alpha}.\ \meta{\tau}}$. Since $\meta{\beta}\in FTV(\forall\meta{\alpha}.\ \meta{\tau})$, according to the definition of $FTV$, $\meta{\beta}\in FTV(\meta{\tau})-\set{\meta{\alpha}}$. By monotonicity, $\meta{\beta}\in FTV(\meta{\tau})$. According to the induction hypothesis, $\meta{\Delta}, \meta{\alpha}\vdash \meta{\beta}$. Since according to the definition of $FTV$, $\meta{\beta} \neq \meta{\alpha}$, therefore, according to \textit{Lemma 0}, $\meta{\Delta}\vdash\meta{\beta}$.
    \end{itemize}
  \end{proof}
  \begin{quote}
     \textbf{Lemma 1-1: forall $\meta{\Delta}$, $\meta{\Gamma}$, $\meta{\alpha}$, $\meta{\beta}$, $\meta{e}$, if $\meta{\beta} \in FTV(\meta{\alpha})$ and $\meta{\Delta};\meta{\Gamma}\vdash\meta{e}:\meta{\alpha}$ then $\meta{\Delta}\vdash\meta{\beta}$}
  \end{quote}
  \begin{proof}
    By induction on $\meta{\Delta};\meta{\Gamma}\vdash\meta{e}:\meta{\alpha}$.
    \begin{itemize}
      \item Case $\inferrule{
        \meta{x} \in \dom{\meta{\Gamma}}\and
        \meta{\Gamma}(\meta{x}) = \meta{\alpha}\and
        \meta{\Delta}\vdash\meta{\Gamma}
      }{\meta{\Delta};\meta{\Gamma}\vdash\meta{x}:\meta{\alpha}}$. In this case, $\meta{\alpha}$ is a type variable, and $FTV(\meta{\alpha}) = \meta{\alpha}$, thus $\meta{\beta}=\meta{\alpha}$. Since $\meta{\Delta}\vdash\meta{\Gamma}$ and $\meta{\Gamma}(\meta{x})=\meta{\alpha}$, $\meta{\Delta}\vdash\meta{\alpha}$ and thus $\meta{\Delta}\vdash\meta{\beta}$
      \item Case $\inferrule{\meta{\Delta}\vdash\meta{\tau_1}\and \meta{\Delta};\meta{\Gamma}[\meta{x}\mapsto\meta{\tau_1}]\vdash\meta{e} : \meta{\tau_2}}{\meta{\Delta};\meta{\Gamma}\vdash\lambda\meta{x}:\meta{\tau}.\,\meta{e} : \meta{\tau_1}\to\meta{\tau_2}}$. According to the definition of $FTV$, $\meta{\beta}\in FTV(\meta{\tau_1}) \cup FTV(\meta{\tau_2})$. By case split on the occurence of $\meta{\beta}$.
        \begin{enumerate}
          \item Case $\meta{\beta}\in FTV(\meta{\tau_1})$. According to \textit{Lemma 1-0}, $\meta{\Delta}\vdash\meta{\beta}$.
          \item Case $\meta{\beta}\in FTV(\meta{\tau_2})$. According to the induction hypothesis and \textit{Lemma 1-0}, $\meta{\Delta}\vdash\meta{\beta}$.
        \end{enumerate}
      \item Case $\inferrule{
        \meta{\Delta};\meta{\Gamma}\vdash\meta{e_1}:\meta{\tau_1}\to\meta{\tau_2}\and
        \meta{\Delta};\meta{\Gamma}\vdash\meta{e_2}:\meta{\tau_1}
      }{\meta{\Delta};\meta{\Gamma}\vdash\meta{e_1}\ \meta{e_2}:\meta{\tau_2}}$. Since $\meta{\beta}\in FTV(\meta{\tau_2})$, $\meta{\beta}\in FTV(\meta{\tau_1})\cup FTV(\meta{\tau_2})$. Therefore, according to the the induction hypothesis, $\meta{\Delta}\vdash\meta{\beta}$.
      \item Case $\inferrule{
        \meta{\Delta};\meta{\Gamma}\vdash\meta{e}:\forall\meta{\alpha}.\ \meta{\tau}
      }{\meta{\Delta};\meta{\Gamma}\vdash\meta{e}\ \meta{\tau_1}:\meta{\tau[\meta{\tau_1}/\meta{\alpha}]}}$. Since there is no $\meta{\alpha}$ in $\meta{\tau}[\meta{\tau_1}/\meta{\alpha}]$, so $\meta{\beta}\neq \meta{\alpha}$. Moreover, $\meta{\beta}\in FTV(\meta{\tau[\meta{\tau_1}/\meta{\alpha}]})$, so directly we can get $\meta{\beta}\in FTV(\meta{\tau}) - \set{\meta{\alpha}}$. Thus according to the induction hypothesis, $\meta{\Delta}\vdash \meta{\beta}$.
    \end{itemize}
  \end{proof}
  \begin{quote}
    If $\cdot;\cdot\vdash\meta{e}:\meta{\tau}$, then $FTV(\meta{\tau}) = \emptyset$.
  \end{quote}
  \begin{proof}
    Suppose $FTV(\meta{\tau}) \neq \emptyset$, then $\exists \meta{\beta} \in FTV(\meta{\tau})$. According to $\textit{Lemma 1-1}$, $\cdot\vdash\meta{\beta}$. However, $\meta{\Delta}$ in this case is an empty set, this contradicts with $\cdot\vdash\meta{\beta}$. Therefore, $FTV(\meta{\tau})$ must be $\emptyset$.
  \end{proof}

\item Consider the following pseudo untyped $\lambda$-calculus program, which assumes a language with built-in integers, booleans, and pairs.
  \[
    \lambda f.\ (f\ 0, f\ \mathtt{true})
  \]
  \begin{enumerate}[(a),left=1em]
  \item Suppose you transcribed this program into OCaml (no need to turn in any
    such transcription). Explain briefly and informally why it would not
    typecheck. (It's ok to base your explanation purely on your intuition about
    how the OCaml type system works, not on any formal system.)
  \item\label{principle:first} Show how to transcribe (by adding type
    annotations, type abstractions, and type applications) this program into
    System F (assume you have built-in integers, booleans, and pairs) such that
    it has the following type.
    \[
      (\forall\alpha.\, \alpha\to\alpha)\to int\times bool
    \]
    No need to prove formally that your transcription has this type. Just convince yourself.
  \item Using your understanding of parametricity, say what the ``only''
    possible value to pass for $f$ is in your transcription in part~\ref{principle:first}.
    No need to prove your answer.
  \item\label{principle:second} Show how to \emph{again} transcribe (by adding type
    annotations, type abstractions, and type applications) this program into
    System F (assume you have built-in integers, booleans, and pairs) such that
    it has the following (different!) type.
    \[
      \forall\alpha.\, (\forall\beta.\, \beta\to\alpha)\to \alpha\times\alpha
    \]
    No need to prove formally that your transcription has this type. Just convince yourself.
  \item\label{principle:second-f} Using your understanding of parametricity, describe the possible values
    to pass in for $f$ in your transcription form part~\ref{principle:second}.
    No need to prove your answer.

    Hint: There are infinitely many, but they all have a clean description.
  \item Given your answer to part~\ref{principle:second-f}, what can you say
    about the pair returned by the System F program from part~\ref{principle:second}?
    No need to prove your answer.
  \end{enumerate}
  As an aside, the examples in this problem demonstrate the lack of ``principle
  types'' for System F. A principle type for an expression is its most general
  type, in the sense that if it has any other type, then it is a special case of
  its principle type. Principle types exist in ML, but not in System F, as
  demonstrated by this problem. The lack of principle types poses a serious
  difficulty to type inference, because it means there is no ``best answer'' to
  return for the type of an expression.

\clearpage
\item We will use a dot ``$\cdot$'' to represent an empty partial function for
  the $\meta{\rho}$ argument to $R$.
  \begin{enumerate}[(a),left=1em]
  \item Translate the meaning of $R_{\forall\alpha.\, \alpha\to\alpha}^{\cdot}$ into English. (You can use symbols in your English.)
  \item
  Show directly from the definition of $R$ that
  \[
    \Lambda\alpha.\,\lambda x:\alpha.\, x \quad\in\quad R_{\forall\alpha.\, \alpha\to\alpha}^{\cdot}
  \]
  \end{enumerate}
\item This question is about the definition of $R$ itself, and specifically its
  ``presupposition''.  A presupposition is kind of like a precondition, but on a
  mathematical object instead of a program.  It means that the mathematical
  object doesn't make sense unless the presupposition is true. According to the
  last page of this document, the presupposition of
  $R_{\meta{\tau}}^{\meta{\rho}}$ is that
  $FTV(\meta{\tau})\subseteq\dom{\meta{\rho}}$.
  \begin{enumerate}[(a),left=1em]
  \item In the base case of the definition of $R$, when looking at a type
    variable $\meta{\alpha}$, we look up the type variable in
    $\meta{\rho}$. Since $\meta{\rho}$ is a partial function, this only makes
    sense if $\meta{\alpha}\in\dom{\meta{\rho}}$.  Prove in one short sentence
    that the presupposition of $R$ guarantees
    $\meta{\alpha}\in\dom{\meta{\rho}}$.
  \item Since $R$ is defined by recursion on $\meta{\tau}$, we should technically check
    that any recursive calls to $R$ satisfy their presupposition,
    \emph{assuming} the presupposition of the ``outer'' $R$. There are three
    recursive calls in the definition of $R$. Prove that each of them satisfy
    the presupposition.
  \end{enumerate}
\item In Homework~3 (programming part) we saw how to Church-encode pairs in the
  untyped $\lambda$-calculus, as follows
\begin{verbatim}
    pair = \x. \y. \f. f x y
    fst = \p. p (\x. \y. x)
\end{verbatim}
  This encoding can be typed in System F as follows. The type of pairs whose
  first components have type $\meta{\tau_1}$ and whose second components have
  type $\meta{\tau_2}$ will be \emph{abbreviated}
  $Pair\ \meta{\tau_1}\ \meta{\tau_2}$, which is defined as follows:
  \[
    Pair\ \meta{\tau_1}\ \meta{\tau_2} = \forall\alpha.\, (\meta{\tau_1}\to\meta{\tau_2}\to\alpha)\to\alpha.
  \]
  \begin{enumerate}[(a),left=1em]
  \item
    The type of $\mathtt{pair}$ is then
    \[
      \forall\alpha.\,\forall\beta.\, \alpha\to\beta\to Pair\ \alpha\ \beta,
    \]
    or, expanding the definition of $Pair$,
    \[
      \forall\alpha.\,\forall\beta.\, \alpha\to\beta\to \forall\gamma.\, (\alpha\to\beta\to\gamma)\to\gamma.
    \]

    Show how to transcribe the untyped program \texttt{pair} from Homework~3
    given above into System F (by adding type annotations, type abstractions,
    and type applications) such that it has the above type. No need to formally
    prove it has the type. Just convince yourself.
  \item
    Similarly, the type of $\mathtt{fst}$ is then
    \[
      \forall\alpha.\,\forall\beta.\, Pair\ \alpha\ \beta \to \alpha,
    \]
    or, expanding the definition of $Pair$,
    \[
      \forall\alpha.\,\forall\beta.\, (\forall\gamma.\, (\alpha\to\beta\to\gamma)\to\gamma) \to \alpha,
    \]
    Show how to transcribe the untyped program \texttt{fst} from Homework~3
    given above into System F (by adding type annotations, type abstractions,
    and type applications) such that it has the above type. No need to formally
    prove it has the type. Just convince yourself.
  \item Prove directly using the operational semantics that, for any values
    $\meta{v_1}:\meta{\tau_1}$ and $\meta{v_2}:\meta{\tau_2}$,
    \[
      \mathtt{fst}\ \meta{\tau_1}\ \meta{\tau_2}\ (\mathtt{pair}\ \meta{\tau_1}\ \meta{\tau_2}\ \meta{v_1}\ \meta{v_2}) \to^* \meta{v_1}.
    \]
    where \texttt{fst} and \texttt{pair} refer to your transcribed versions in System F.
  \item Now suppose $\meta{p}$ is \emph{any} System F expression such that
    \[
      \cdot;\cdot\vdash \meta{p}:\forall\alpha.\,\forall\beta.\, \alpha\to\beta\to \forall\gamma.\, (\alpha\to\beta\to\gamma)\to\gamma.
    \]
    In other words, $\meta{p}$ is just some program with the same type as
    \texttt{pair}.  Similarly, suppose that $\meta{f}$ is some System F
    expression such that
    \[
      \cdot;\cdot\vdash \meta{f}:\forall\alpha.\,\forall\beta.\, (\forall\gamma.\, (\alpha\to\beta\to\gamma)\to\gamma) \to \alpha.
    \]
    In other words, $\meta{f}$ has the same type as \texttt{fst}.

    Use the fundamental theorem of the logical relation to prove that, for any values
    $\meta{v_1}:\meta{\tau_1}$ and $\meta{v_2}:\meta{\tau_2}$,
    \[
      \meta{f}\ \meta{\tau_1}\ \meta{\tau_2}\ (\meta{p}\ \meta{\tau_1}\ \meta{\tau_2}\ \meta{v_1}\ \meta{v_2}) \to^* \meta{v_1}.
    \]
  \end{enumerate}
\end{enumerate}

\noindent{}The remaining problem is extra credit.

\begin{enumerate}[resume*]
\item This extra credit problem considers adding a new expression to System F, called \texttt{choose}.
  The idea is that $\mathtt{choose}\ \meta{v_1}\ \meta{v_2}$ nondeterministically evaluates to
  either $\meta{v_1}$ or $\meta{v_2}$.
  \begin{enumerate}[(a),left=1em]
  \item Give operational semantics for \texttt{choose} that first evaluate its
    first argument to a value, then evaluate its second argument to a value, and
    then \emph{either} evaluate to the first value \emph{or} the second.

    Hint: Use four rules. One to make recursive progress on the first argument,
    and a similar one for the second argument. Then one to ``choose'' the first
    value, and one to ``choose'' the second.
  \item Give a typing rule for \texttt{choose}.

    Hint: Use one rule. It is vaguely similar to \texttt{if}, except there is no
    branch condition.
  \item\label{choose:bool} Use \texttt{choose} to define a Church boolean that is not ``equivalent'' to true or false,
    in the sense that can return either its first argument or its second, and change its mind each time it's called.
  \item Prove that the fundamental theorem of the (unary) logical relation still holds
    on this extended language extending the proof with a case for \texttt{choose}.
  \item Explain how the existence of your program from part~\ref{choose:bool}
    does \emph{not} contradict the result we proved on slide 13 of Lecture 15 about
    Church booleans using the (unary) logical relation.
  \item Extra extra credit (requires bonus material from Lecture 15 on binary logical relations).
    Explain how the existence of your program from part~\ref{choose:bool}
    \emph{does} contradict the result we proved on slide 17 of Lecture 15 about
    Church booleans using the \emph{binary} logical relation.
  \item Extra extra credit (requires bonus material from Lecture 15 on binary logical
    relations).  Attempt to prove the case for \texttt{choose} in the
    fundamental theorem of the \emph{binary} logical relation. Point to exactly
    where you get stuck.
  \end{enumerate}
\end{enumerate}

\clearpage
  \noindent System F\hfill{}Syntax\hspace{1cm}\hfill{}
  \[
    \begin{array}{rcl}
      \meta{e} & ::= & \meta{x} \mid \meta{e}\ \meta{e} \mid \lambda\meta{x}:\meta{\tau}.\,\meta{e} \mid \Lambda\meta{\alpha}.\,\meta{e} \mid \meta{e}\ \meta{\tau}\\
      \meta{v} & ::= & \lambda\meta{x}.\,\meta{e} \mid \Lambda\meta{\alpha}.\,\meta{e}\\
      \meta{\tau} & ::= & \meta{\alpha} \mid \meta{\tau}\to\meta{\tau} \mid \forall\meta{\alpha}.\meta{\tau} \\
      \meta{\Gamma} & \in & Var \rightharpoonup Type\\
      \meta{\Delta} & \subseteq & TyVar

    \end{array}
  \]
  \boxed{\meta{e}\to\meta{e}}\hfill{}Operational Semantics\hspace{1cm}\hfill{}
  \begin{mathpar}
    \inferrule{\meta{e_1}\to\meta{e_1'}}{\meta{e_1}\ \meta{e_2}\to\meta{e_1'}\ \meta{e_2}}
    \and
    \inferrule{\meta{e_2}\to\meta{e_2'}}{\meta{v_1}\ \meta{e_2}\to\meta{v_1}\ \meta{e_2'}}
    \and
    \inferrule{ }{(\lambda\meta{x}:\meta{\tau}.\,\meta{e})\ \meta{v}\to \meta{e}[\meta{v}/\meta{x}]}
    \and
    \inferrule{\meta{e}\to\meta{e'}}{\meta{e}\ \meta{\tau}\to\meta{e'}\ \meta{\tau}}
    \and
    \inferrule{ }{(\Lambda\meta{\alpha}.\,\meta{e})\ \meta{\tau} \to \meta{e}[\meta{\tau}/\meta{\alpha}]}
  \end{mathpar}
  \boxed{\meta{e}\to^*\meta{e}}\vspace{-7mm}
  \begin{mathpar}
    \inferrule{ }{\meta{e}\to^*\meta{e}}
    \and
    \inferrule{\meta{e_1}\to^*\meta{e_2}\and\meta{e_2}\to\meta{e_3}}{\meta{e_1}\to^*\meta{e_3}}
  \end{mathpar}
  \noindent\boxed{\meta{e}[\meta{e_1}/\meta{x}]}\hfill{}Substitution functions\hspace{1cm}\hfill{}
  \[
    \begin{array}{rcll}
      \meta{x}[\meta{e_1}/\meta{x}] & = & \meta{e_1} & \\
      \meta{y}[\meta{e_1}/\meta{x}] & = & \meta{y} & \qquad \meta{y} \ne \meta{x}\\
      (\meta{e_2}\ \meta{e_3})[\meta{e_1}/\meta{x}] & = & \meta{e_2}[\meta{e_1}/\meta{x}]\ \meta{e_3}[\meta{e_1}/\meta{x}]&\\
      (\lambda\meta{y}:\meta{\tau}.\,\meta{e})[\meta{e_1}/\meta{x}] & = & \lambda\meta{y}:\meta{\tau}.\,\meta{e}[\meta{e_1}/\meta{x}]& \qquad \meta{y}\ne\meta{x}\text{ and }\meta{y}\not\in FV(\meta{e_1})\\
      (\meta{e}\ \meta{\tau})[\meta{e_1}/\meta{x}] & = & \meta{e}[\meta{e_1}/\meta{x}]\ \meta{\tau}&\\
      (\Lambda\meta{\alpha}.\,\meta{e})[\meta{e_1}/\meta{x}] & = & \Lambda\meta{\alpha}.\,\meta{e}[\meta{e_1}/\meta{x}]&\qquad\meta{\alpha}\not\in FTV(\meta{e_1})
    \end{array}
  \]
  \noindent\boxed{\meta{e}[\meta{\tau}/\meta{\alpha}]}
  \[
    \begin{array}{rcll}
      \meta{x}[\meta{\tau}/\meta{\alpha}] & = & \meta{e_1} & \\
      (\meta{e_2}\ \meta{e_3})[\meta{\tau}/\meta{\alpha}] & = & \meta{e_2}[\meta{\tau}/\meta{\alpha}]\ \meta{e_3}[\meta{\tau}/\meta{\alpha}]&\\
      (\lambda\meta{x}:\meta{\tau_1}.\,\meta{e})[\meta{\tau}/\meta{\alpha}] & = & \lambda\meta{x}:\meta{\tau_1}[\meta{\tau}/\meta{\alpha}].\,\meta{e}[\meta{\tau}/\meta{\alpha}]& \\
      (\meta{e}\ \meta{\tau_1})[\meta{\tau}/\meta{\alpha}] & = & \meta{e}[\meta{\tau}/\meta{\alpha}]\ \meta{\tau_1}[\meta{\tau}/\meta{\alpha}]&\\
      (\Lambda\meta{\beta}.\,\meta{e})[\meta{\tau}/\meta{\alpha}] & = & \Lambda\meta{\beta}.\,\meta{e}[\meta{\tau}/\meta{\alpha}]& \qquad \meta{\beta}\ne\meta{\alpha}\text{ and }\meta{\beta}\not\in FTV(\meta{\tau})
    \end{array}
  \]
  \noindent\boxed{\meta{\tau}[\meta{\tau_1}/\meta{\alpha}]}
  \[
    \begin{array}{rcll}
      \meta{\alpha}[\meta{\tau_1}/\meta{\alpha}] & = & \meta{\tau_1} & \\
      \meta{\beta}[\meta{\tau_1}/\meta{\alpha}] & = & \meta{\beta} & \qquad\meta{\beta}\ne\meta{\alpha}\\
      (\meta{\tau_2}\to\meta{\tau_3})[\meta{\tau_1}/\meta{\alpha}] & = & \meta{\tau_2}[\meta{\tau_1}/\meta{\alpha}]\to\meta{\tau_3}[\meta{\tau_1}/\meta{\alpha}] & \\
      (\forall\meta{\beta}.\,\meta{\tau})[\meta{\tau_1}/\meta{\alpha}] & = & \forall\meta{\beta}.\,\meta{\tau}[\meta{\tau_1}/\meta{\alpha}] & \qquad \meta{\beta}\ne\meta{\alpha}\text{ and }\meta{\beta}\not\in FTV(\meta{\tau_1})
    \end{array}
  \]
  \noindent\boxed{FTV(\meta{\tau})}\hfill{}Free type variables of a type or expression\hspace{1cm}\hfill{}
  \[
    \begin{array}{rcl}
      FTV(\meta{\alpha}) & = & \set{\meta{\alpha}}\\
      FTV(\meta{\tau_1}\to\meta{\tau_2}) & = & FTV(\meta{\tau_1}) \cup FTV(\meta{\tau_2})\\
      FTV(\forall\meta{\alpha}.\,\meta{\tau}) & = & FTV(\meta{\tau}) - \set{\meta{\alpha}}
    \end{array}
  \]
  \noindent\boxed{FTV(\meta{e})}\hfill{}Note that we overload $FTV$ on expressions and types.\hspace{8mm}\hfill{}
  \[
    \begin{array}{rcl}
      FTV(\meta{x}) & = & \emptyset\\
      FTV(\lambda\meta{x}:\meta{\tau}.\,\meta{e}) & = & FTV(\meta{\tau}) \cup FTV(\meta{e})\\
      FTV(\meta{e_1}\ \meta{e_2}) & = & FTV(\meta{e_1}) \cup FTV(\meta{e_2})\\
      FTV(\Lambda\meta{\alpha}.\,\meta{e}) & = & FTV(\meta{e}) - \set{\meta{\alpha}}\\
      FTV(\meta{e}\ \meta{\tau}) & = & FTV(\meta{e}) \cup FTV(\meta{\tau})
    \end{array}
  \]
  \noindent\boxed{FV(\meta{e})}\hfill{}Free variables of an expression\hspace{1cm}\hfill{}
  \[
    \begin{array}{rcll}
      FV(\meta{x}) & = & \set{\meta{x}}\\
      FV(\lambda\meta{x}:\meta{\tau}.\,\meta{e}) & = & FV(\meta{e}) - \set{\meta{x}}\\
      FV(\meta{e_1}\ \meta{e_2}) & = & FV(\meta{e_1}) \cup FV(\meta{e_2})\\
      FV(\Lambda\meta{\alpha}.\,\meta{e}) & = & FV(\meta{e})\\
      FV(\meta{e}\ \meta{\tau}) & = & FV(\meta{e})

    \end{array}
  \]
  \noindent\boxed{\meta{\Delta};\meta{\Gamma}\vdash\meta{e}:\meta{\tau}}\hfill{}Type System\hspace{2cm}\hfill{}
  \begin{mathpar}
    \inferrule{\meta{x}\in\dom{\meta{\Gamma}} \and \meta{\Gamma}(\meta{x}) = \meta{\tau}}{\meta{\Delta};\meta{\Gamma}\vdash\meta{x}:\meta{\tau}}\\
    \and
    \inferrule{\meta{\Delta}\vdash\meta{\tau_1}\and \meta{\Delta};\meta{\Gamma}[\meta{x}\mapsto\meta{\tau_1}]\vdash\meta{e} : \meta{\tau_2}}{\meta{\Delta};\meta{\Gamma}\vdash\lambda\meta{x}:\meta{\tau}.\,\meta{e} : \meta{\tau_1}\to\meta{\tau_2}}
    \and
    \inferrule{\meta{\Delta};\meta{\Gamma}\vdash\meta{e_1} : \meta{\tau_1}\to\meta{\tau_2}\and\meta{\Delta};\meta{\Gamma}\vdash\meta{e_2} : \meta{\tau_1}}{\meta{\Delta};\meta{\Gamma}\vdash\meta{e_1}\ \meta{e_2} : \meta{\tau_2}}\\
    \and
    \inferrule{\meta{\alpha}\not\in\meta{\Delta}\and\meta{\Delta},\meta{\alpha};\meta{\Gamma}\vdash\meta{e}:\meta{\tau}}{\meta{\Delta};\meta{\Gamma}\vdash\Lambda\meta{\alpha}.\, \meta{e}:\forall\meta{\alpha}.\,\meta{\tau}}
    \and
    \inferrule{\meta{\Delta}\vdash\meta{\tau_1}\and\meta{\Delta};\meta{\Gamma}\vdash\meta{e}:\forall\meta{\alpha}.\,\meta{\tau}}{\meta{\Delta};\meta{\Gamma}\vdash\meta{e}\ \meta{\tau_1}:\meta{\tau}[\meta{\tau_1}/\meta{\alpha}]}
  \end{mathpar}
  \boxed{\meta{\Delta}\vdash\meta{\Gamma}}\vspace{-3mm}
  \[
    \meta{\Delta}\vdash\meta{\Gamma} = \forall\meta{x}\in\dom{\meta{\Gamma}}.\ \meta{\Delta}\vdash\meta{\Gamma}(\meta{x})
  \]
  \boxed{\meta{\Delta}\vdash\meta{\tau}}\vspace{-7mm}
  \begin{mathpar}
    \inferrule{\meta{\alpha}\in\meta{\Delta}}{\meta{\Delta}\vdash\meta{\alpha}}
    \and
    \inferrule{\meta{\Delta}\vdash\meta{\tau_1}\and\meta{\Delta}\vdash\meta{\tau_2}}{\meta{\Delta}\vdash\meta{\tau_1}\to\meta{\tau_2}}
    \and
    \inferrule{\meta{\Delta},\meta{\alpha}\vdash\meta{\tau}}{\meta{\Delta}\vdash\forall\meta{\alpha}.\,\meta{\tau}}
  \end{mathpar}
  \hfill{}Preparation for the definition of the logical relation\hfill{}
  \[
    \begin{array}{rcl}
      Spec & = & \set{\meta{S} \subseteq Val \mid \forall\meta{e}\in \meta{S}.\,\meta{e}\text{ closed}}\\
      \meta{\rho} & \in & TyVar \rightharpoonup Spec\\
      \meta{\gamma} & \in & Var \rightharpoonup Val\\
      T(\meta{S}) & = & \set{\meta{e} \mid\exists{\meta{v}}.\, \meta{e}\to^*\meta{v} \wedge \meta{v}\in\meta{S}}
    \end{array}
  \]
  \hfill{}Definition of the logical relation on closed terms\hfill{}

  \noindent\boxed{R_{\meta{\tau}}^{\meta{\rho}}} $\text{presupposes }FTV(\meta{\tau}) \subseteq\dom{\meta{\rho}}$
  \[
    \begin{array}{rcl}
      R_{\meta{\alpha}}^{\meta{\rho}} & = & \meta{\rho}(\meta{\alpha})\\
      R_{\meta{\tau_1}\to\meta{\tau_2}}^{\meta{\rho}} & = & \set{\lambda\meta{x}.\,\meta{e} \mid \forall\meta{v}\in R_{\meta{\tau_1}}^{\meta{\rho}}.\ \meta{e}[\meta{x}/\meta{v}]\in T(R_{\meta{\tau_2}}^{\meta{\rho}})}\\
      R_{\forall\meta{\alpha}.\,\meta{\tau}}^{\meta{\rho}} & = & \set{\Lambda\meta{\alpha}.\,\meta{e} \mid \forall\meta{S}\in Spec.\, \meta{e}\in T(R_{\meta{\tau}}^{\meta{\rho}[\meta{\alpha}\mapsto\meta{S}]})}\\
    \end{array}
  \]
  \noindent\boxed{\meta{e}[\meta{\gamma}]}\hfill{}Multisubstitution\hfill{}
  \[
    \begin{array}{rcll}
      \meta{x}[\meta{\gamma}] & = & \meta{\gamma}(\meta{x}) & \qquad\text{if } \meta{x}\in\dom{\meta{\gamma}}\\
      \meta{x}[\meta{\gamma}] & = & \meta{x} & \qquad\text{if } \meta{x}\not\in\dom{\meta{\gamma}}\\

      (\meta{e_2}\ \meta{e_3})[\meta{\gamma}] & = & \meta{e_2}[\meta{\gamma}]\ \meta{e_3}[\meta{\gamma}]&\\
      (\lambda\meta{x}:\meta{\tau}.\,\meta{e})[\meta{\gamma}] & = & \lambda\meta{x}:\meta{\tau}.\,\meta{e}[\meta{\gamma}]& \qquad \meta{x}\not\in\dom{\meta{\gamma}}\text{ and }\forall\meta{y}\in\dom{\meta{\gamma}}.\,\meta{x}\not\in FV(\meta{\gamma}(\meta{y}))\\
      (\meta{e}\ \meta{\tau})[\meta{\gamma}] & = & \meta{e}[\meta{\gamma}]\ \meta{\tau}&\\
      (\Lambda\meta{\alpha}.\,\meta{e})[\meta{\gamma}] & = & \Lambda\meta{\alpha}.\,\meta{e}[\meta{\gamma}]&\qquad
\forall\meta{x}\in\dom{\meta{\gamma}}.\,\meta{\alpha}\not\in FTV(\meta{\gamma}(\meta{x}))
    \end{array}
  \]
  \hfill{}Preparation and definition of the open logical relation\hfill{}

  \noindent\boxed{\meta{\rho};\meta{\Gamma}\vdash\meta{\gamma}}\vspace{-3mm}
  \[
    \meta{\rho};\meta{\Gamma}\vdash\meta{\gamma} =
    \forall\meta{x}\in\dom{\meta{\Gamma}}.\
    \meta{x}\in\dom{\meta{\gamma}} \wedge \meta{\gamma}(\meta{x})\in R_{\meta{\Gamma}(\meta{x})}^{\meta{\rho}}
  \]
  \boxed{\meta{\Delta};\meta{\Gamma}\vDash\meta{e}:\meta{\tau}}\vspace{-3mm}
  \[
    \meta{\Delta};\meta{\Gamma}\vDash\meta{e}:\meta{\tau} \quad=\quad
    \forall\meta{\rho}.\
    \dom{\meta{\Delta}}\subseteq \dom{\meta{\rho}}\Rightarrow
    \forall\meta{\gamma}.\
    \meta{\rho};\meta{\Gamma}\vdash\meta{\gamma}\Rightarrow
    \meta{e}[\meta{\gamma}]\in T(R_{\meta{\tau}}^{\meta{\rho}})
  \]
  \begin{theorem}[Fundamental theorem of the logical relation]
    If $\meta{\Delta};\meta{\Gamma}\vdash\meta{e}:\meta{\tau}$ then
    $\meta{\Delta};\meta{\Gamma}\vDash\meta{e}:\meta{\tau}$.
  \end{theorem}
\end{document}