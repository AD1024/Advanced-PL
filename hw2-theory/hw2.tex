\documentclass{article}

\usepackage{lmodern}
\usepackage{concrete}
\usepackage{eulervm}
\usepackage[T1]{fontenc}
\usepackage{amsmath,amsthm,amssymb}
\usepackage{mathpartir}
\usepackage{stmaryrd}
\usepackage{xcolor}
\usepackage[shortlabels]{enumitem}
\usepackage{listings}

% copied from package braket
% edited to remove whitespace inside braces
% and to simplify the definition of \Set because we don't use double-vert
{\catcode`\|=\active
  \xdef\set{\protect\expandafter\noexpand\csname set \endcsname}
  \expandafter\gdef\csname set \endcsname#1{\mathinner
    {\lbrace{\mathcode`\|32768\let|\midvert #1}\rbrace}}
  \xdef\Set{\protect\expandafter\noexpand\csname Set \endcsname}
  \expandafter\gdef\csname Set \endcsname#1{\left\{%
     {\mathcode`\|32768\let|\SetVert #1}\right\}}
}
\def\midvert{\egroup\mid\bgroup}
\makeatletter
\def\mid@vertical{\mskip1mu\vrule\mskip1mu}
\def\SetVert{{\egroup\;\mid@vertical\;\bgroup}}
\makeatother


\lstdefinelanguage{imp}
{morekeywords={while,assert,skip,failed},
sensitive=true,
morecomment=[l]{//},
morestring=[b]",
}
\lstset{
language=imp,
basicstyle=\ttfamily,
keywordstyle=\bfseries,
identifierstyle=,
showstringspaces=false}


\newtheorem{theorem}{Theorem}

\usepackage[margin=1in]{geometry}
\usepackage{changepage}

\usepackage{fancyhdr}
\pagestyle{fancy}
\fancyhead[C]{Homework 2: Theory part}
\fancyhead[R]{CSE 490P}
\setlength{\headheight}{14pt}
\newcommand{\meta}[1]{{\color{blue}#1}}
\newcommand{\imp}{\textsc{Imp}}

%\skip is already a LaTeX command, so we pick another name
\newcommand{\impskip}{\ensuremath{\text{\ttfamily\bfseries skip}}}
\newcommand{\assert}{\ensuremath{\text{\ttfamily\bfseries assert}}}
\newcommand{\failed}{\ensuremath{\text{\ttfamily\bfseries failed}}}
\newcommand{\while}{\ensuremath{\text{\ttfamily\bfseries while}}}
\newcommand{\assign}{\ensuremath{\mathrel{\text{\texttt{:=}}}}}

\begin{document}

\begin{enumerate}[leftmargin=*,itemindent=*,start=1,label={{\bf Problem \arabic*}.},ref=\arabic*]
\item\label{odeds-program} Consider the following \imp{} program.
  \begin{lstlisting}
    a := 0;
    b := 0;
    c := 0;
    while true {
      assert a >= 0;
      a := a + b;
      b := b + c;
      c := c + 1;
    }
  \end{lstlisting}
  \begin{enumerate}[(a)]
  \item Manually convert this program into a transition system, where each step
    of the transition system corresponds to one iteration of the body of the
    loop. Use some ``poetic license'' in your conversion, so that the resulting
    transition system is as simple as possible. (Do not model the first three
    lines of the program as transitions, but rather as part of your definition
    of the initial states. Also, ignore the \lstinline|assert| for now.)

    Be sure to clearly define the three components of your transition system:
    state space, initial states, and step relation.
  \item Convert the assertion into a property $P$ of states in your transition
    system. Explain briefly and informally why $P$ is an invariant of your
    transition system.
  \item Explain why a direct proof that $P$ is an invariant by induction on
    executions fails. (Don't just say ``it fails because it's not
    inductive''. Either walk through the start of a proof and point to exactly
    where you get stuck because the induction hypothesis is not strong
    enough, or give a concrete counterexample to induction (CTI) that shows $P$
    is not inductive and explain why your CTI is, in fact, a CTI.)
  \item Prove $P$ is an invariant indirectly: find a property $I$ that implies
    $P$ and prove $I$ is an invariant directly by induction on executions. Be
    sure to clearly state your definition of $I$. Try to make your $I$ as simple
    as possible.
  \item Does your invariant $I$ capture exactly the set of reachable states in
    your transition system?
    \begin{itemize}
    \item If so, prove it by proving that for every state satisfying $I$, you
      can construct an execution that reaches it.
    \item If not, give an example of a state that satisfies $I$ but is not
      reachable. Prove that your example state is not reachable. (How do we show
      a state is not reachable...?)
    \end{itemize}
    In your answer, be sure to clearly state which of the two possibilities
    above you are proving. Since there are many invariants $I$ that are
    inductive and imply $P$, there are many different answers to this
    question. That's ok! Just answer the question for the $I$ you picked in the
    previous part.
  \end{enumerate}
  \clearpage
\item This problem is about treating the small-step operational semantics of a
  language as a transition system.
  \begin{enumerate}[(a)]
  \item The small-step operational semantics for \imp{} are themselves already a
    giant transition system whose state space consists of all possible pairs of
    heaps and statements and whose transitions are just steps in the small-step
    operational semantics.

    Define a set of initial states that describes the states at the beginning of
    an execution of the program from Problem~\ref{odeds-program}. Try to make
    your definition of the initial states as simple as possible. (Think
    carefully about what constraints, if any, you need to place on the heap.)
  \item\label{reach-stmts} A state in this transition system is a pair of a heap and a (possibly
    complex) statement. Describe the set of \emph{statements} that can appear in
    reachable states, ignoring the heap. No need to prove that your answer is correct just yet.
    (Hint: There are more than 5 and fewer than 15.
    It's ok not to list them out explicitly, but to describe them succinctly.)
  \item Are there finitely many \emph{heaps} that can appear in reachable states?
    Give a one-sentence proof of your yes/no answer.
  \item Using the ``official'' version of the semantics for \imp{} (i.e., the
    one with \failed{}, which is reproduced at the end of this document), we can
    encode the English statement
    \begin{quote}
      The assertion in the program from Problem~\ref{odeds-program} is never
      violated.
    \end{quote}
    by saying that the statement \failed{} does not appear in any reachable
    state. State this formally using metavariables and symbols. Call this
    property $P$.
  \item Explain informally why $P$ is an invariant of your transition system.
  \item Explain why a direct proof that $P$ is an invariant by induction on
    executions fails. (Don't just say ``it fails because it's not inductive''.
    Either walk through the start of a proof and point to exactly
    where you get stuck because the induction hypothesis is not strong
    enough, or, give a concrete CTI and explain why your CTI is, in fact, a CTI.)
  \item\label{opsem-ind-proof} Prove $P$ is an invariant indirectly: find a property $I$ that implies
    $P$ and prove $I$ is an invariant directly by induction on executions. Be
    sure to clearly state your definition of $I$.
  \item Explain why your $I$ in this problem is more complicated than the $I$
    from Problem~\ref{odeds-program}. (Hint: What is the ``granularity'' of a
    step in the transition system from Problem~\ref{odeds-program}? What about
    in the transition system in this problem? What is the state space in
    Problem~\ref{odeds-program}? What about in this problem?)
  \item Does your invariant $I$ capture exactly the set of \emph{statements} that
    can appear in reachable states, as you enumerated in part~\ref{reach-stmts}?
    \begin{itemize}
    \item If so, prove it by proving that for every statement that can appear in
      a state satisfying $I$, you can construct an execution that reaches a state
      containing that statement.
    \item If not, give an example of a statement that can appear in a state
      satisfying $I$ but not in any reachable state. Prove that your example
      statement cannot appear in any reachable state. (How do we show something
      about all reachable states...?)
    \end{itemize}
    In your answer, be sure to clearly state which of the two possibilities
    above you are proving. Since there are many invariants $I$ that are
    inductive and imply $P$, there are many different answers to this
    question. That's ok! Just answer the question for the $I$ you picked in
    part~\ref{opsem-ind-proof}.
  \end{enumerate}
\clearpage
\item Consider the transition system from lecture that we produced by manually
  translating our favorite program into a transition system. We reproduce the
  definition here. The transition system is parameterized by an integer $n$.
  \begin{align*}
    S &= \set{(x,y) | x,y\in \mathbb{Z}}\\
    S_0 &= \set{(0, n)}\\
    \to &= \Set{\big((x,y),(x+1,y-1)\big) | y > 0}
  \end{align*}
  \begin{enumerate}[(a)]
  \item Which states cannot step? Give your answer as a definition of a set of
    states, and a one-sentence proof that a state cannot step if and only if it
    is in your set.
    $$ S = \set{(x, y) | x, y \in \mathbb{Z} \wedge y \leq 0}$$
    \textbf{States in this set cannot step, because according to our definition of step, a state can step if and only if $y>0$. However, in this set of states, the $y$ values are less or equal to 0 thus cannot step.}
  \item Define a set of ``final'' states, which are those we intuitively expect
    execution to end in. There is more than one reasonable answer here, so pick
    whatever seems most intuitive to you, subject to your ability to prove the
    next part below.
    $$ S = \set{(x, 0)| x \in \mathbb{Z} \wedge x \geq 0}$$
  \item Prove that for all $n\in\mathbb{Z}$, if $n\ge 0$ then there exists a
    final state $s$ such that $(0,n) \to^* s$. (Hint: There is more than one
    thing you could try to prove by induction, but really only one choice of
    \emph{what} to induct on.)\\
    \textbf{Lemma 1: for all $x, y\in \mathbb{N}$, $(x, y) \to^* s$ for some arbitary but fixed final state s}
    \begin{proof}
      By induction on $y$.
      \begin{itemize}
        \item $y = 0$. In this case, by the reflexivity of step and $\to^*$, $(x, 0) \to^* (x, 0)$.
        \item $y = y^\prime + 1$. According to the definition of step, $(x, y^\prime + 1) \to (x + 1, y^\prime)$. Since $x \in \mathbb{N}$ and $y^\prime \in \mathbb{N}$, $x + 1\in \mathbb{N}$, and thus according to the induction hypothesis, $(x + 1, y^\prime) \to^* s$ for some arbitary but fixed final state $s$. According to the definition of $\to^*$, 
        $$\inferrule{(x, y^\prime + 1) \to (x + 1, y^\prime)\and (x + 1, y^\prime) \to^* s}{(x, y^\prime + 1) \to^* s}$$
        for some arbitary but fixed final state $s$.
      \end{itemize}
    \end{proof}

    \textbf{Theorem: for all $n\in\mathbb{Z}$, if $n\ge 0$ then there exists a
    final state $s$ such that $(0,n) \to^* s$}
    \begin{proof}
      Since $0 \in\mathbb{N}$, and $n \in\mathbb{Z} \wedge n \geq 0 \leftrightarrow n \in\mathbb{N}$, according to \textit{Lemma 1}, $(0, n) \to^* s$ for some fixed final state s.
    \end{proof}
  \item Explain how your proof from the previous part, together with the fact
    that the transition system is deterministic, also shows that there are no
    infinite executions in this transition system.

    \textbf{According to the proof above, for all initial state $s \in \mathbb{S}_0$, there exists a final state $s^\prime$ such that $s\to^*s^\prime$. Since this transition system is deterministic, there exists exactly one such final state $s^\prime$ that $s\to^*s^\prime$. So for all $s\in\mathbb{S}_0$, the execution starts at $s$ terminates at a unique final state $s^\prime$.}

%     \textbf{We can classify states to two categories: $\mathbb{S}^+$ denotes a set of states that are able to make at least one step, and $\mathbb{S}^-$ which denotes states that are not able to step. According to the definition of step in this system, it is obvious that (only consider the case where $x, y \in \mathbb{Z}$)
%     $$(x, y) \in \mathbb{S}^+ \Leftrightarrow x \geq 0 \wedge y > 0$$ Since $\mathbb{S}^-$ is disjoint with $\mathbb{S}^+$, therefore apparently $$(x, y)\in \mathbb{S}^- \Leftrightarrow x < 0 \vee y \leq 0$$} We can define $\to^*$ as a function $f : \mathbb{S}^+ \cup \mathbb{S}^- \to \mathbb{S}^+ \cup \mathbb{S}^-$, such that
%     \begin{align*}
%       &\forall (x, y)
%  \in \mathbb{S}^+, f(x, y) = f (x + 1, y - 1) & \text{Takes n steps}\\
%       & \forall (x, y) \in \mathbb{S}^-, f(x, y) = (x, y) & \text{Takes 0 step}
%     \end{align*}
%     According to the proof in previous part, since this transition system is deterministic, for all admissible inputs $(x, y) \in \mathbb{S}_0$, there exists a \textit{unique} final state $s \in \mathbb{S}^-$ such that $(x, y) \to^* s$, which means $f(x, y) = s$. Apparently, $f(f(x, y)) = f(x, y)$, which means function $f$ has a fixpoint, which the program terminates at.
  \item Construct an infinite execution of your transition system from
    Problem~\ref{odeds-program}.

    \textbf{The transition system is}
    \begin{align*}
      S &= \set{(a,b,c) | x,y\in \mathbb{Z}}\\
      S_0 &= \set{(0, 0, 0)}\\
      \to &= \Set{\big((a,b,c),(a+b,b+c,c+1)\big) | a \geq 0}
    \end{align*}
    Frist, prove that the transition system does have infinite execution by showing that $a \geq 0$ holds for all reachable states. We choose a strenghened invariant: $a, b, c\in \mathbb{Z} \wedge a, b, c \geq 0$.
    \begin{proof}
      By induction on the states.
      \begin{enumerate}
        \item $s=(0, 0, 0)$. The initial state satisfies the condition that $a, b, c\in \mathbb{Z} \wedge a, b, c \geq 0$.
        \item $(a, b, c)\to (a^\prime=a+b, b^\prime=b+c, c^\prime=c+1)$. The induction hypothesis tells us that $a, b, c\in \mathbb{Z} \wedge a, b, c \geq 0$. Therefore, $a^\prime\geq a+b\wedge b^\prime\geq b+c \wedge c^\prime > c$. Therefore, $a^\prime, b^\prime, c^\prime \in \mathbb{Z}\wedge a^\prime \geq 0 \wedge b^\prime\geq 0\wedge c^\prime\geq 0$.
      \end{enumerate}
    \end{proof}
    The infinite execution is:
    $$(0, 0, 0)\to(0, 0, 1)\to(0, 1, 2)\to(1, 3, 3)\to(4, 6, 4)\to(10, 10, 5)\cdots$$

  \end{enumerate}
\item Consider the ``official'' version of \imp{}, with \failed{} (full
  semantics on next page). In Lecture~6, we stated the following progress lemma:
  \begin{quote}
    If $\vdash_{\meta{\Sigma}} \meta{s}\ ok$ and
    $\vdash_{\meta{\Sigma}} \meta{H}\ ok$, then $\meta{s} = \impskip$ or
    $\meta{s} = \failed$ or there exists $\meta{H'},\meta{s'}$ such that
    $\meta{H},\meta{s} \to \meta{H'},\meta{s'}$
  \end{quote}
  \begin{enumerate}[(a)]
  \item Enumerate all the things you \emph{could} try to induct on. For each
    one, say whether you think the proof would work if you inducted on that thing.
    (As usual, it's ok to be wrong.)
  \item Prove the progress lemma by induction on a thing of your choice. As we
    mentioned in class, you will need at least one lemma. Try to make your
    lemma(s) as simple as possible.  You do not need to prove any additional
    lemmas other than progress, just state what you're using clearly (and
    convince yourself it's true!).
  \end{enumerate}

\end{enumerate}

\clearpage
  \imp
  \[
    \begin{array}{rcl}
      \meta{e} & ::= & \meta{n} \mid \meta{e} + \meta{e} \mid \meta{e} - \meta{e}
                       \mid \meta{b} \mid \meta{e} < \meta{e} \mid \meta{e} = \meta{e} \mid \meta{x}\\
      \meta{v} & ::= & \meta{n} \mid \meta{b}\\
      \meta{s} & ::= & \impskip \mid \assert\ \meta{e} \mid \failed \mid \meta{x} \assign \meta{e} \mid \meta{s}; \meta{s} \mid \while\ \meta{e}\ \meta{s}\\
      \meta{\tau} & ::= & int \mid bool
    \end{array}
    \qquad
    \begin{array}{rcl}
      \meta{n} & \in & \mathbb{Z}\\
      \meta{b} & \in & \mathbb{B}\\
      \meta{x} & \in & \mathrm{Var}\ (= \mathrm{String})\\
      \meta{H} & \in & \mathrm{Var} \rightharpoonup \mathrm{Value}\\
      \meta{\Sigma} & \in & \mathrm{Var} \rightharpoonup \mathrm{Type}
    \end{array}
  \]
  \boxed{\meta{H},\meta{e} \Downarrow \meta{v}}\vspace{-7mm}
  \begin{mathpar}
    \inferrule{ }{\meta{H},\meta{n}\Downarrow\meta{n}}
    \and
    \inferrule{ }{\meta{H},\meta{b}\Downarrow\meta{b}}
    \and
    \inferrule{x \in \mathrm{dom}\ \meta{H} \and \meta{H}(\meta{x}) = \meta{v}}{\meta{H},\meta{x}\Downarrow\meta{v}}\\
    \and
    \inferrule{\meta{H},\meta{e_1}\Downarrow\meta{n_1} \and \meta{H},\meta{e_2}\Downarrow\meta{n_2}}{\meta{H},\meta{e_1} + \meta{e_2}\Downarrow\meta{n_1 + n_2}}
    \and
    \inferrule{\meta{H},\meta{e_1}\Downarrow\meta{n_1} \and \meta{H},\meta{e_2}\Downarrow\meta{n_2}}{\meta{H},\meta{e_1} - \meta{e_2}\Downarrow\meta{n_1 - n_2}}\\
    \and
    \inferrule{\meta{H},\meta{e_1}\Downarrow\meta{n_1} \and \meta{H},\meta{e_2}\Downarrow\meta{n_2}}{\meta{H},\meta{e_1} < \meta{e_2}\Downarrow\meta{n_1 < n_2}}
    \and
    \inferrule{\meta{H},\meta{e_1}\Downarrow\meta{v_1} \and \meta{H},\meta{e_2}\Downarrow\meta{v_2}}{\meta{H},\meta{e_1} = \meta{e_2}\Downarrow\meta{n_1 = n_2}}
  \end{mathpar}
  \boxed{\meta{H},\meta{s} \to \meta{H},\meta{s}}\vspace{-3mm}
  \begin{mathpar}
    \inferrule{\meta{H},\meta{e}\Downarrow\meta{v}}{\meta{H},\meta{x}\assign\meta{e} \to \meta{H}[\meta{x}\mapsto\meta{v}],\impskip}
    \and
    \inferrule{\meta{H},\meta{e}\Downarrow\meta{\top}}{\meta{H},\assert\ \meta{e} \to \meta{H},\impskip}
    \and
    \inferrule{\meta{H},\meta{e}\Downarrow\meta{\bot}}{\meta{H},\assert\ \meta{e} \to \meta{H},\failed}
    \and
    \inferrule{\meta{H},\meta{s_1} \to \meta{H'},\meta{s_1'}}{\meta{H},\meta{s_1};\meta{s_2} \to \meta{H'},\meta{s_1'};\meta{s_2}}
    \and
    \inferrule{ }{\meta{H},\impskip;\meta{s} \to \meta{H},\meta{s}}
    \and
    \inferrule{ }{\meta{H},\failed;\meta{s} \to \meta{H},\failed}\\
    \and
    \inferrule{\meta{H},\meta{e}\Downarrow\meta{\top}}{\meta{H},\while\ \meta{e}\ \meta{s} \to \meta{H},\meta{s};\while\ \meta{e}\ \meta{s}}
    \and
    \inferrule{\meta{H},\meta{e}\Downarrow\meta{\bot}}{\meta{H},\while\ \meta{e}\ \meta{s} \to \meta{H},\impskip}
  \end{mathpar}
  \boxed{\vdash_{\meta{\Sigma}} \meta{e} : \meta{\tau}}\vspace{-7mm}
  \begin{mathpar}
    \inferrule{ }{\vdash_{\meta{\Sigma}} \meta{n} : int}
    \and
    \inferrule{ }{\vdash_{\meta{\Sigma}} \meta{b} : bool}
    \and
    \inferrule{x \in \mathrm{dom}\ \meta{\Sigma} \and \meta{\Sigma}(\meta{x}) = \meta{\tau}}{\vdash_{\meta{\Sigma}} \meta{x} : \meta{\tau}}\\
    \and
    \inferrule{ \vdash_{\meta{\Sigma}} \meta{e_1} : int \\ \vdash_{\meta{\Sigma}} \meta{e_2} : int }{\vdash_{\meta{\Sigma}} \meta{e_1} + \meta{e_2} : int}
    \and
    \inferrule{ \vdash_{\meta{\Sigma}} \meta{e_1} : int \\ \vdash_{\meta{\Sigma}} \meta{e_2} : int }{\vdash_{\meta{\Sigma}} \meta{e_1} - \meta{e_2} : int}\\
    \and
    \inferrule{ \vdash_{\meta{\Sigma}} \meta{e_1} : int \\ \vdash_{\meta{\Sigma}} \meta{e_2} : int }{\vdash_{\meta{\Sigma}} \meta{e_1} < \meta{e_2} : bool}
    \and
    \inferrule{ \vdash_{\meta{\Sigma}} \meta{e_1} : \meta{\tau} \\ \vdash_{\meta{\Sigma}} \meta{e_2} : \meta{\tau} }
              {\vdash_{\meta{\Sigma}} \meta{e_1} = \meta{e_2} : bool}
  \end{mathpar}
  \boxed{\vdash_{\meta{\Sigma}} \meta{s}\ ok}\vspace{-5mm}
  \begin{mathpar}
    \inferrule{ }{\vdash_{\meta{\Sigma}} \impskip\ ok}
    \and
    \inferrule{x \in \mathrm{dom}\ \meta{\Sigma} \and  \vdash_{\meta{\Sigma}} \meta{e} : \meta{\Sigma}(\meta{x})}{\vdash_{\meta{\Sigma}} \meta{x}\assign\meta{e}\ ok}
    \and
    \inferrule{\vdash_{\meta{\Sigma}} \meta{e} : bool}{\vdash_{\meta{\Sigma}} \assert\ \meta{e}\ ok}\\
    \and
    \inferrule{\vdash_{\meta{\Sigma}} \meta{s_1}\ ok \and \vdash_{\meta{\Sigma}} \meta{s_2}\ ok}{\vdash_{\meta{\Sigma}} \meta{s_1};\meta{s_2}\ ok}
    \and
    \inferrule{\vdash_{\meta{\Sigma}} \meta{e} : bool \and \vdash_{\meta{\Sigma}} \meta{s}\ ok}{\vdash_{\meta{\Sigma}} \while\ \meta{e}\ \meta{s}\ ok}
  \end{mathpar}
  \boxed{\vdash_{\meta{\Sigma}} \meta{H}\ ok}\vspace{-5mm}
  \begin{mathpar}
    \inferrule{\forall \meta{x}\in\mathrm{dom}\ \meta{\Sigma},\ \meta{x}\in\mathrm{dom}\ \meta{H} ~\wedge{}
                 \vdash_{\meta{\Sigma}} \meta{H}(\meta{x}) : \meta{\Sigma}(\meta{x})}
              {\vdash_{\meta{\Sigma}} \meta{H}\ ok}
  \end{mathpar}
\end{document}